\documentclass[a4paper]{article}
\usepackage[ngerman]{babel}
\usepackage[T1]{fontenc}
\usepackage[latin1]{inputenc}
\usepackage{lmodern}
\usepackage[top=2.5cm, bottom=2.5cm, left=2.5cm, right=2cm]{geometry}
\usepackage[final]{pdfpages}
\usepackage{graphicx}
\usepackage[colorlinks=true, urlcolor=blue]{hyperref}
\usepackage{multirow} 
\usepackage[headsepline,footsepline]{scrpage2} %[headtopline,headsepline,footsepline,footbotline]
\pagestyle{scrheadings}


\title{%
  \textsc{Projekt 2}\\[2ex]
  \textsc{Protokoll}}
\author{%
  \textsc{Team1}}
\date{%
  \textsc{}}

\clearscrheadfoot

%\tiny
%\scriptsize
%\footnotesize 
%\small 
%\normalsize 
%\large 
%\Large 
%\LARGE 
%\huge 
%\Huge 

\begin{document}
%Kopfzeile
\ihead{\includegraphics[scale=1]{fhnw_ht_logo_de}}
%\chead{Kopfzeile Mitte}
%Datum anpassen***************************************************************
\ohead{Protkoll vom 01.04.2015} 

%Fusszeile
\ifoot{Projekt 2}
\cfoot{Team 1}
\ofoot{\pagemark}

%Titel
%Datum anpassen**************************************************************
\section*{\textbf{Protokoll zur Sitzung am 01.04.2015}}

%F�r normale Schriftgr�sse
\normalsize
\begin{center}
\begin{tabular}{|l|l l|}
\hline
%Datum, Zeit, Ort anpassen*************
Datum, Zeit, Ort & \multicolumn{2}{|l|}{01.04.2015, 15:00-16:00 Uhr, FHNW Windisch, Geb�ude 1, 3. Etage}\\ \hline
Besprechungsleiter & Alexander Stocker & EIT Student / Projektleiter \\ \hline

%Anzahl Anwesende bestimmt die Zahl von multirow**********
\multirow{6}{*}{Teilnehmer} & Alexander Stocker & EIT Student / Projektleiter \\
& Claudius J�rg & EIT Student / Projektleiter Stellvertreter \\
& Martin Moser & EIT Student / Fachspezialist Technik \\
& Reto Freivogel & EIT Student / Fachspezialist Java \\
& Denis Stampfli & EIT Student / Java \\
& Yohannes Measho & EIT Student / Technik \\ \hline

%NICHT VERAENDERN*******************
\multirow{6}{*}{Verteiler} & Alexander Stocker & \href{mailto:alexander.stocker@students.fhnw.ch}{alexander.stocker@students.fhnw.ch} \\
& Claudius J�rg & \href{mailto:claudius.joerg@students.fhnw.ch}{claudius.joerg@students.fhnw.ch} \\
& Martin Moser & \href{mailto:martin.moser1@students.fhnw.ch}{martin.moser1@students.fhnw.ch} \\
& Reto Freivogel & \href{mailto:reto.freivogel@students.fhnw.ch}{reto.freivogel@students.fhnw.ch} \\
& Denis Stampfli & \href{mailto:denis.stampfli@students.fhnw.ch}{denis.stampfli@students.fhnw.ch} \\
& Yohannes Measho & \href{mailto:yohannes.measho@students.fhnw.ch}{yohannes.measho@students.fhnw.ch} \\ \hline

Sitzungsziele & \multicolumn{2}{|l|}{Siehe Traktandenliste}\\ \hline

%Anzahl Traktanden bestimmt die Zahl von multirow**********
\multirow{7}{*}{Traktandenliste} & \multicolumn{2}{|l|}{Begr�ssung}\\
& \multicolumn{2}{|l|}{Administratives}\\
& \multicolumn{2}{|l|}{Aktueller Stand}\\
& \multicolumn{2}{|l|}{Kurzfeedback besprechen}\\
& \multicolumn{2}{|l|}{Weiteres Vorgehen / Ausblick}\\
& \multicolumn{2}{|l|}{Fragen und Anregungen}\\
& \multicolumn{2}{|l|}{Einf�hrung in GitHub}\\
\hline

Protokollf�hrer & Claudius J�rg & EIT Student \\ \hline
\end{tabular}\newline
\end{center}
\noindent\rule{\textwidth}{0.4pt}
\subsection*{\textbf{Begr�ssung}}
\subsubsection*{\tiny{DISKUSSION}}

Der Sitzungsleiter begr�sst alle Teilnehmer und stellt kurz die Traktandenliste vor.

\noindent\rule{\textwidth}{0.4pt}
\subsection*{\textbf{Administratives}}%**********
\subsubsection*{\tiny{DISKUSSION}}

Mail von Sonntag von Denis hat PL entt�uscht. Eine Aufgabe wurde beschlossen und ist darum Pflichtbewusst zu erledigen.
Bei Konflikten offen kommunizieren und L�sungsans�tze zeigen. Es wird nicht eigenm�chtig �ber den Auftrag entschieden.
\newline
Es wurde ein Bundesordner mit den wichtigsten Projektdokumenten erstellt.
\newline
Im Protokoll m�ssen die Beschl�sse und Auftr�ge klarer getrennt werden.
\newline
Werden Dateien auf Sharepoint nicht synchronisiert, wird als Loesung die Dateiendung mit underline OK erg�nzt.

\subsubsection*{\tiny{BESCHLUSS}}


\noindent\rule{\textwidth}{0.4pt}
\subsection*{\textbf{Aktueller Stand}}%**********
\subsubsection*{\tiny{DISKUSSION}}

Inhaltsverzeichnis des Fachberichtes soll bis nach Osterferien erstellt werden.
\newline
Alle M-Files von Martin sind auskommentiert (exkl. Sani.m)
\newline
Layout wurde nicht gemacht, wird im Traktandum Ausblick behandelt.
\newline
Java Fortschritt: Klassen (Regler, -strecke, Kreis- und Helferklassen) sind geschrieben und Matlab kann aufgerufen werden.
Matlab Fortschritt: Phasengangmethode gibt diverse Diagramme aus.

\subsubsection*{\tiny{BESCHLUSS}}



\noindent\rule{\textwidth}{0.4pt}
\subsection*{\textbf{Kurzfeedback besprechen}}
\subsubsection*{\tiny{DISKUSSION}}

Die Formeln im Pflichtenheft wurden bem�ngelt, sie haben Fehler.

\subsubsection*{\tiny{BESCHLUSS}}

Formeln vor der Ver�ffentlichung pr�fen

\noindent\rule{\textwidth}{0.4pt}
\subsection*{\textbf{Weiteres Vorgehen / Ausblick}}
\subsubsection*{\tiny{DISKUSSION}}

Statusbericht bis Sonntag an Buchschacher senden.
\newline
Layout muss wieder mit Niklaus besprochen werden.

\subsubsection*{\tiny{BESCHLUSS}}

\begin{description}
\item[Alex]
\begin{itemize}
\item Statusbericht erstellen bis Sonntag, 05.04.15.
\item Layout zeichnen in Java, 15.04.15
\item Erstellt Zwischenpr�sentation (10 Folien), 08.04.15.
\item Kl�rt ab wo die Zwischenpr�sentation stattfindet.
\end{itemize}
\item[Claudius]
\begin{itemize}
\item Kostentracking in Statusbericht erarbeiten, bis Sonntag, 05.04.15.
\item Statusberichte �bernehmen bis 02.04.15.
\end{itemize}
\item[Denis] Termin mit Niklaus vereinbaren und Layout besprechen (evtl. kann das Reto im Unterricht �bernehmen), Dienstag,
14.04.15.
\item[Martin]
\begin{itemize}
\item Befasst sich mit Befehl SPLINE im Matlab, Mittwoch, 15.04.15.
\item Erstellt Folien f�r technischen Teil den Zwischenpr�sentation bis 08.04.2015
\end{itemize}
\item[Reto]
\begin{itemize}
\item Zwischenpr�sentation: Folien erstellen zu Frage: Wo im Programm was passiert, Mittwoch, 08.04.15.
\item Java programmieren: Matlab aufrufen, Eingabe/Ausgabe von Daten. Mittwoch, 15.04.15.
\end{itemize}
\item[Yohannes] Adaptiert das Inhaltsverzeichnis des Fachberichtbeispiels auf Projekt und sendet das Alexander, Freitag, 10.04.15.
\end{description}

\noindent\rule{\textwidth}{0.4pt}
\subsection*{\textbf{Fragen und Anregungen}}
\subsubsection*{\tiny{DISKUSSION}}

Keine

\subsubsection*{\tiny{BESCHLUSS}}

Keine

\noindent\rule{\textwidth}{0.4pt}
\subsection*{\textbf{Einf�hrung in GitHub}}
\subsubsection*{\tiny{DISKUSSION}}

Reto erkl�rt dem Team die Bedienung von GitHub.

\subsubsection*{\tiny{BESCHLUSS}}

\noindent\rule{\textwidth}{0.4pt}

\vfill

\begin{tabbing}
tab1 \= tab2 \= tab3 \= tab4 \= tab5 \= tab6 \= tab7 \= tab8 \= tab9 \= tab10 \= tab11 \= tab12 \= tab13 \= tab14 \kill

\>\>\> Besprechungsleiter \>\>\>\>\>\>\>\>\>\> Protokollf�hrer\\
\>\>\> Alexander Stocker \>\>\>\>\>\>\>\>\>\> Claudius J�rg\\
\>\>\> 01.04.2015 \>\>\>\>\>\>\>\>\>\> 01.04.2015\\
\end{tabbing}

%\begin{description}
%\item[Alex] Analyse Schrittantworten
%\item[Claudius] Rechercheordner auf GI erstellen, Thema: Streckenidentifikation
%\item[Denis] Thema: Amplitudengang
%\item[Martin] Analyse Schrittantworten, Thema: Phasengang
%\item[Reto] Layout Entwurf von GUI, Analyse Schrittantworten, Thema: Phasengangmethode/Frequenzsimulation
%\item[Yohannes] Thema: Mathematische �bertragungsfunktion
%\end{description}

\end{document}