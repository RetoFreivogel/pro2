\documentclass[a4paper]{article}
\usepackage[ngerman]{babel}
\usepackage[T1]{fontenc}
\usepackage[latin1]{inputenc}
\usepackage{lmodern}
\usepackage[top=2.5cm, bottom=2.5cm, left=2.5cm, right=2cm]{geometry}
\usepackage[final]{pdfpages}
\usepackage{graphicx}
\usepackage[colorlinks=true, urlcolor=blue]{hyperref}
\usepackage{multirow} 
\usepackage[headsepline,footsepline]{scrpage2} %[headtopline,headsepline,footsepline,footbotline]
\pagestyle{scrheadings}


\title{%
  \textsc{Projekt 2}\\[2ex]
  \textsc{Protokoll}}
\author{%
  \textsc{Team1}}
\date{%
  \textsc{}}

\clearscrheadfoot

%\tiny
%\scriptsize
%\footnotesize 
%\small 
%\normalsize 
%\large 
%\Large 
%\LARGE 
%\huge 
%\Huge 

\begin{document}
%Kopfzeile
\ihead{\includegraphics[scale=1]{fhnw_ht_logo_de}}
%\chead{Kopfzeile Mitte}
%Datum anpassen***************************************************************
\ohead{Protkoll vom 25.02.2015} 

%Fusszeile
\ifoot{Projekt 2}
\cfoot{Team 1}
\ofoot{\pagemark}

%Titel
%Datum anpassen**************************************************************
\section*{\textbf{Protokoll zur Sitzung am 25.02.2015}}

%F�r normale Schriftgr�sse
\normalsize
\begin{center}
\begin{tabular}{|l|l l|}
\hline
%Datum, Zeit, Ort anpassen*************
Datum, Zeit, Ort & \multicolumn{2}{|l|}{25.02.2015, 15:30-16:15 Uhr, FHNW Windisch, Geb�ude 1, 3. Etage}\\ \hline
Besprechungsleiter & Alexander Stocker & EIT Student / Projektleiter \\ \hline

%Anzahl Anwesende bestimmt die Zahl von multirow**********
\multirow{6}{*}{Teilnehmer} & Alexander Stocker & EIT Student / Projektleiter \\
& Claudius J�rg & EIT Student / Projektleiter Stellvertreter \\
& Martin Moser & EIT Student / Fachspezialist Technik \\
& Reto Freivogel & EIT Student / Fachspezialist Java \\
& Denis Stampfli & EIT Student / Java \\
& Yohannes Measho & EIT Student / Technik \\ \hline

%Anzahl Anwesende bestimmt die Zahl von multirow**********
\multirow{6}{*}{Verteiler} & Alexander Stocker & \href{mailto:alexander.stocker@students.fhnw.ch}{alexander.stocker@students.fhnw.ch} \\
& Claudius J�rg & \href{mailto:claudius.joerg@students.fhnw.ch}{claudius.joerg@students.fhnw.ch} \\
& Martin Moser & \href{mailto:martin.moser1@students.fhnw.ch}{martin.moser1@students.fhnw.ch} \\
& Reto Freivogel & \href{mailto:reto.freivogel@students.fhnw.ch}{reto.freivogel@students.fhnw.ch} \\
& Denis Stampfli & \href{mailto:denis.stampfli@students.fhnw.ch}{denis.stampfli@students.fhnw.ch} \\
& Yohannes Measho & \href{mailto:yohannes.measho@students.fhnw.ch}{yohannes.measho@students.fhnw.ch} \\ \hline

Sitzungsziele & \multicolumn{2}{|l|}{Siehe Traktandenliste}\\ \hline

%Anzahl Traktanden bestimmt die Zahl von multirow**********
\multirow{4}{*}{Traktandenliste} & \multicolumn{2}{|l|}{Begr�ssung}\\
& \multicolumn{2}{|l|}{Arbeitspakete definieren}\\
& \multicolumn{2}{|l|}{Technische Anforderungen}\\
& \multicolumn{2}{|l|}{Softwarestruktur}\\ \hline

Protokollf�hrer & Claudius J�rg & EIT Student \\ \hline
\end{tabular}\newline
\end{center}
\noindent\rule{\textwidth}{0.4pt}
\subsection*{\textbf{Begr�ssung}}
\subsubsection*{\tiny{DISKUSSION}}

Der Sitzungsleiter begr�sst alle Teilnehmer und stellt kurz die Traktandenliste vor.

\noindent\rule{\textwidth}{0.4pt}
\subsection*{\textbf{Arbeitspakete definieren}}
\subsubsection*{\tiny{DISKUSSION}}

PL stellt ersten Entwurf von PSP vor. F�r die Phasengangmethode steht nur das Dokument von P. Niklaus z.V.
Vorschlag das technische zu unterteilen, weil grosses Thema. Kurze Erkl�rung des Projektauftrages gem�ss Notizen Moser.
Direkte �bertragung in PSP von Stocker, Grundlage f�r Recherchearbeit. Stocker erl�utert Ablauf bis Abgabe Pflichtenheft.

\subsubsection*{\tiny{BESCHLUSS}}

Jeder schreibt ein �berblick �ber sein Thema um es dem Rest des Teams vorzutragen.

\begin{description}
\item[Alex] Analyse Schrittantworten
\item[Claudius] Rechercheordner auf GI erstellen, Thema: Streckenidentifikation
\item[Denis] Thema: Amplitudengang
\item[Martin] Analyse Schrittantworten, Thema: Phasengang
\item[Reto] Layout Entwurf von GUI, Analyse Schrittantworten, Thema: Phasengangmethode/Frequenzsimulation
\item[Yohannes] Thema: Mathematische �bertragungsfunktion
\end{description}

\noindent\rule{\textwidth}{0.4pt}
\subsection*{\textbf{Technische Anforderungen}}
\subsubsection*{\tiny{DISKUSSION}}

Technische Details wurden in einem ersten Fachinput an Moser, Measho, Freivogel und Stampfli gegeben.Reto hat mit
MATLAB experimentiert, mit Schrittantworten, die Werte stimmen noch nicht.Nur PI und PID-Regler. Erkl�rung in Fachinput 2.
Werte f�r Programm werden grafisch von Dokument ermittelt.
Ziel und mathematischer Vorgang werden besprochen.\\
Ziel: Schrittantwort simulieren von Regler.Der mathematische Weg ist das wichtigste.

\subsubsection*{\tiny{BESCHLUSS}}

Vertiefung in den einzelnen Gebieten gem�ss Aufgaben in ?Arbeitspakete definieren?

\noindent\rule{\textwidth}{0.4pt}
\subsection*{\textbf{Softwarestruktur}}
\subsubsection*{\tiny{DISKUSSION}}

Moser schl�gt Dropdown-Menu vor f�r Faustformeln. Klassendiagramm und Struktur ist zurzeit nicht m�glich. Freivogel
arbeitet laufend an GUI. Erster Entwurf ist auf Blatt. Einfaches Flussdiagramm f�r Pflichtenheft w�nschenswert.

\subsubsection*{\tiny{BESCHLUSS}}

Thema vertagt.

\noindent\rule{\textwidth}{0.4pt}

\vfill

\begin{tabbing}
tab1 \= tab2 \= tab3 \= tab4 \= tab5 \= tab6 \= tab7 \= tab8 \= tab9 \= tab10 \= tab11 \= tab12 \= tab13 \= tab14 \kill

\>\>\> Besprechungsleiter \>\>\>\>\>\>\>\>\>\> Protokollf�hrer\\
\>\>\> Alexander Stocker \>\>\>\>\>\>\>\>\>\> Claudius J�rg\\
\>\>\> 25.02.2015 \>\>\>\>\>\>\>\>\>\> 25.02.2015\\
\end{tabbing}

\end{document}

