\documentclass[a4paper]{article}
\usepackage[ngerman]{babel}
\usepackage[T1]{fontenc}
\usepackage[latin1]{inputenc}
\usepackage{lmodern}
\usepackage[top=2.5cm, bottom=2.5cm, left=2.5cm, right=2cm]{geometry}
\usepackage[final]{pdfpages}
\usepackage{graphicx}
\usepackage[colorlinks=true, urlcolor=blue]{hyperref}
\usepackage{multirow} 
\usepackage[headsepline,footsepline]{scrpage2} %[headtopline,headsepline,footsepline,footbotline]
\pagestyle{scrheadings}


\title{%
  \textsc{Projekt 2}\\[2ex]
  \textsc{Protokoll}}
\author{%
  \textsc{Team1}}
\date{%
  \textsc{}}

\clearscrheadfoot

%\tiny
%\scriptsize
%\footnotesize 
%\small 
%\normalsize 
%\large 
%\Large 
%\LARGE 
%\huge 
%\Huge 

\begin{document}
%Kopfzeile
\ihead{\includegraphics[scale=1]{fhnw_ht_logo_de}}
%\chead{Kopfzeile Mitte}
%Datum anpassen**************************************************************
\ohead{Protokoll vom 03.06.2015} 

%Fusszeile
\ifoot{Projekt 2}
\cfoot{Team 1}
\ofoot{\pagemark}

%Titel
%Datum anpassen**************************************************************
\section*{\textbf{Protokoll zur Sitzung am 03.06.2015}}

%F�r normale Schriftgr�sse
\normalsize
\begin{center}
\begin{tabular}{|l|l l|}
\hline
%Datum, Zeit, Ort anpassen*************
Datum, Zeit, Ort & \multicolumn{2}{|l|}{03.06.2015, 13:15-13:25 Uhr, FHNW Windisch, Geb�ude 1, 3. Etage}\\ \hline
Besprechungsleiter & Alexander Stocker & EIT Student / Projektleiter \\ \hline

%Anzahl Anwesende bestimmt die Zahl von multirow**********
\multirow{6}{*}{Teilnehmer} & Alexander Stocker & EIT Student / Projektleiter \\
& Claudius J�rg & EIT Student / Projektleiter Stellvertreter \\
& Martin Moser & EIT Student / Fachspezialist Technik \\
& Reto Freivogel & EIT Student / Fachspezialist Java \\
& Denis Stampfli & EIT Student / Java \\
& Yohannes Measho & EIT Student / Technik \\ \hline

%NICHT VERAENDERN*******************
\multirow{6}{*}{Verteiler} & Alexander Stocker & \href{mailto:alexander.stocker@students.fhnw.ch}{alexander.stocker@students.fhnw.ch} \\
& Claudius J�rg & \href{mailto:claudius.joerg@students.fhnw.ch}{claudius.joerg@students.fhnw.ch} \\
& Martin Moser & \href{mailto:martin.moser1@students.fhnw.ch}{martin.moser1@students.fhnw.ch} \\
& Reto Freivogel & \href{mailto:reto.freivogel@students.fhnw.ch}{reto.freivogel@students.fhnw.ch} \\
& Denis Stampfli & \href{mailto:denis.stampfli@students.fhnw.ch}{denis.stampfli@students.fhnw.ch} \\
& Yohannes Measho & \href{mailto:yohannes.measho@students.fhnw.ch}{yohannes.measho@students.fhnw.ch} \\ \hline

Sitzungsziele & \multicolumn{2}{|l|}{Siehe Traktandenliste}\\ \hline

%Anzahl Traktanden bestimmt die Zahl von multirow**********
\multirow{3}{*}{Traktandenliste} & \multicolumn{2}{|l|}{Begr�ssung}\\
& \multicolumn{2}{|l|}{R�ckblick}\\
& \multicolumn{2}{|l|}{Weiteres Vorgehen / Ausblick}\\ \hline

Protokollf�hrer & Claudius J�rg & EIT Student \\ \hline
\end{tabular}\newline
\end{center}
\noindent\rule{\textwidth}{0.4pt}
\subsection*{\textbf{Begr�ssung}}
\subsubsection*{\tiny{DISKUSSION}}

Der Sitzungsleiter begr�sst alle Teilnehmer und stellt kurz die Traktandenliste vor.

\noindent\rule{\textwidth}{0.4pt}
\subsection*{\textbf{R�ckblick}}%**********
\subsubsection*{\tiny{DISKUSSION}}

Der Softwarebericht von Reto ist nicht fertig. (Stand von 13.15 Uhr )Es fehlt noch der Text zu den Klassen und sonstige Erg�nzungen,
er sch�tzt noch 1.5 bis 2 Stunden Aufwand. Bei der Software sind 2 Punkte erledigt beim auskommentieren, bei der Fehlerbehebung sind noch Kleinigkeiten zu erledigen.
(Stand von 15.30 Uhr) Bericht �ber das Diagram (Top-Down) ist fertig. Es sind noch anstehende �nderung zur Beschreibung des Designs f�llig.
\newline
Claudius schreibt noch sein Teil des PMA Berichts fertig und macht sich dann Gedanken zur Pr�sentation.
\newline
Alex muss auch noch sein Teil zum PMA Bericht beenden und die beiden Schlussworte schreiben. Nimmt sie
am Sonntag mit.
\newline
Die Anleitung von Denis ist noch nicht Zufriedenstellend.

\subsubsection*{\tiny{BESCHLUSS}}
Reto arbeitet am Softwarebericht der Freitag Abend fertig sein soll. Wenn schon Texte fertig diese an Martin melden. 
\newline
Alex schreibt die Schlussworte bis Sonntag.
\newline
Denis schreibt die Anleitung zu Ende.
\newline
Claudius f�gt die PMA Reflexionen ins Latex ein.

\noindent\rule{\textwidth}{0.4pt}
\subsection*{\textbf{Weiteres Vorgehen / Ausblick}}
\subsubsection*{\tiny{DISKUSSION}}

Bis Sonntag sollen alle Texte soweit fertig sein, damit sie nur noch �berarbeitet werden m�ssen.
\newline
Bis Sonntag macht sich Claudius Gedanken f�r die Pr�sentation. Sie soll eine Einf�hrung (10'-15') mit kurzer Reflexion des PM beinhalten (u.a. wieso das Wunschziel nicht erreicht wurde) und wie die Software (5') entstand, dann wird Denis mit Yohannes ein Anwendungsbeispiel der Software durchf�hren und es gibt einen kurzen Schlussteil (2'). Yohannes sucht sich daf�r in der Regelungstechnik ein Beispiel aus.

\subsubsection*{\tiny{BESCHLUSS}}
Sonntag, 7. Juni, Abschluss des Projekts. Start: 10 Uhr an der HTNW. 
\newline
\begin{description}
\item[Alex] Schreibt die Schlussworte bis Sonntag.
\item[Claudius]
\begin{enumerate}
	\item Macht ersten Entwurf der Pr�sentation gem�ss Diskussion.
	\item Liest die Anleitung von Denis durch und macht sich Notizen.
\end{enumerate}
\item[Denis] Beendet Anleitung.
\item[Martin] �bertr�gt Texte von Reto in Latex und bringt falls n�tig �nderungen an.
\item[Reto] Schreibt den Softwarebericht bis Freitagabend.
\item[Yohannes] Sucht ein (komisches, trauriges, ernstes, interessantes) Beispiel aus der Regelungstechnik f�r die Pr�sentation.
\end{description}

\noindent\rule{\textwidth}{0.4pt}

\vfill

\begin{tabbing}
tab1 \= tab2 \= tab3 \= tab4 \= tab5 \= tab6 \= tab7 \= tab8 \= tab9 \= tab10 \= tab11 \= tab12 \= tab13 \= tab14 \kill

\>\>\> Besprechungsleiter \>\>\>\>\>\>\>\>\>\> Protokollf�hrer\\
\>\>\> Alexander Stocker \>\>\>\>\>\>\>\>\>\> Claudius J�rg\\
\>\>\> 03.06.2015 \>\>\>\>\>\>\>\>\>\> 03.06.2015\\
\end{tabbing}


\end{document}