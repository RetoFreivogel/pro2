\documentclass[a4paper]{article}
\usepackage[ngerman]{babel}
\usepackage[T1]{fontenc}
\usepackage[latin1]{inputenc}
\usepackage{lmodern}
\usepackage[top=2.5cm, bottom=2.5cm, left=2.5cm, right=2cm]{geometry}
\usepackage[final]{pdfpages}
\usepackage{graphicx}
\usepackage[colorlinks=true, urlcolor=blue]{hyperref}
\usepackage{multirow} 
\usepackage[headsepline,footsepline]{scrpage2} %[headtopline,headsepline,footsepline,footbotline]
\pagestyle{scrheadings}


\title{%
  \textsc{Projekt 2}\\[2ex]
  \textsc{Protokoll}}
\author{%
  \textsc{Team1}}
\date{%
  \textsc{}}

\clearscrheadfoot

%\tiny
%\scriptsize
%\footnotesize 
%\small 
%\normalsize 
%\large 
%\Large 
%\LARGE 
%\huge 
%\Huge 

\begin{document}
%Kopfzeile
\ihead{\includegraphics[scale=1]{fhnw_ht_logo_de}}
%\chead{Kopfzeile Mitte}
%Datum anpassen***************************************************************
\ohead{Protkoll vom 04.03.2015} 

%Fusszeile
\ifoot{Projekt 2}
\cfoot{Team 1}
\ofoot{\pagemark}

%Titel
%Datum anpassen**************************************************************
\section*{\textbf{Protokoll zur Sitzung am 04.03.2015}}

%F�r normale Schriftgr�sse
\normalsize
\begin{center}
\begin{tabular}{|l|l l|}
\hline
%Datum, Zeit, Ort anpassen*************
Datum, Zeit, Ort & \multicolumn{2}{|l|}{04.03.2015, 15:30-16:45 Uhr, FHNW Windisch, 6.2A08}\\ \hline
Besprechungsleiter & Alexander Stocker & EIT Student / Projektleiter \\ \hline

%Anzahl Anwesende bestimmt die Zahl von multirow**********
\multirow{6}{*}{Teilnehmer} & Alexander Stocker & EIT Student / Projektleiter \\
& Claudius J�rg & EIT Student / Projektleiter Stellvertreter \\
& Martin Moser & EIT Student / Fachspezialist Technik \\
& Reto Freivogel & EIT Student / Fachspezialist Java \\
& Denis Stampfli & EIT Student / Java \\
& Yohannes Measho & EIT Student / Technik \\ \hline

%NICHT VERAENDERN*******************
\multirow{6}{*}{Verteiler} & Alexander Stocker & \href{mailto:alexander.stocker@students.fhnw.ch}{alexander.stocker@students.fhnw.ch} \\
& Claudius J�rg & \href{mailto:claudius.joerg@students.fhnw.ch}{claudius.joerg@students.fhnw.ch} \\
& Martin Moser & \href{mailto:martin.moser1@students.fhnw.ch}{martin.moser1@students.fhnw.ch} \\
& Reto Freivogel & \href{mailto:reto.freivogel@students.fhnw.ch}{reto.freivogel@students.fhnw.ch} \\
& Denis Stampfli & \href{mailto:denis.stampfli@students.fhnw.ch}{denis.stampfli@students.fhnw.ch} \\
& Yohannes Measho & \href{mailto:yohannes.measho@students.fhnw.ch}{yohannes.measho@students.fhnw.ch} \\ \hline

Sitzungsziele & \multicolumn{2}{|l|}{Siehe Traktandenliste}\\ \hline

%Anzahl Traktanden bestimmt die Zahl von multirow**********
\multirow{4}{*}{Traktandenliste} & \multicolumn{2}{|l|}{Begr�ssung}\\
& \multicolumn{2}{|l|}{Administratives}\\
& \multicolumn{2}{|l|}{Aktueller Stand}\\
& \multicolumn{2}{|l|}{Weiteres Vorgehen / Ausblick}\\ \hline

Protokollf�hrer & Claudius J�rg & EIT Student \\ \hline
\end{tabular}\newline
\end{center}
\noindent\rule{\textwidth}{0.4pt}
\subsection*{\textbf{Begr�ssung}}
\subsubsection*{\tiny{DISKUSSION}}

Der Sitzungsleiter begr�sst alle Teilnehmer und stellt kurz die Traktandenliste vor.

\noindent\rule{\textwidth}{0.4pt}
\subsection*{\textbf{Administratives}}%**********
\subsubsection*{\tiny{DISKUSSION}}

Jedes Teammitglied berichtet dem PL von den Arbeitsstunden.
\newline
Pflichtenheftvorlagen f�r jede Gruppe auf Sharepoints.
\newline
Inhalte f�r Pflichtenheft an Alexander senden. Am Montag 23.03.15. kann das Pflichtenheft auf dem Sharepoint
gegengelesen werden.

\subsubsection*{\tiny{BESCHLUSS}}

Ganzes Team: Inhalte Pflichtenheft bis Freitag, 20.03.15. an Alexander.

%\begin{description}
%\item[Alex] Analyse Schrittantworten
%\item[Claudius] Rechercheordner auf GI erstellen, Thema: Streckenidentifikation
%\item[Denis] Thema: Amplitudengang
%\item[Martin] Analyse Schrittantworten, Thema: Phasengang
%\item[Reto] Layout Entwurf von GUI, Analyse Schrittantworten, Thema: Phasengangmethode/Frequenzsimulation
%\item[Yohannes] Thema: Mathematische �bertragungsfunktion
%\end{description}

\noindent\rule{\textwidth}{0.4pt}
\subsection*{\textbf{Aktueller Stand}}%**********
\subsubsection*{\tiny{DISKUSSION}}

Zielkatalog wird noch weiter bearbeitet.
\newline
Reto stellt erster Layoutversuch vor. Im Plenum wird die Vorlage f�r die Kundensitzung besprochen und bestimmt.
\newline
Die Recherche l�uft nach wie vor.

\subsubsection*{\tiny{BESCHLUSS}}

\begin{description}
\item[Ganzes Team] macht sich vertraut mit Unterschied PI-PID Regler, Anwendungsgebiet, Auswahl der Regler.
\item[Alexander] Erstellt mit Claudius den Zielkatalog bis Sonntag.
\item[Martin] kl�rt bis zur n�chsten Sitzung ab wie man auf die Schrittantwort der Regelung kommt.
\end{description}

\noindent\rule{\textwidth}{0.4pt}
\subsection*{\textbf{Weiteres Vorgehen / Ausblick}}
\subsubsection*{\tiny{DISKUSSION}}

Das ganze Team nimmt am Kundengespr�ch teil. Fragen f�r Auftraggeber an Alexander.
\newline
Termin und Ort f�r Kundengespr�ch wird noch bekanntgegeben.

\subsubsection*{\tiny{BESCHLUSS}}

\begin{description}
\item[Ganzes Team] Fragen f�r Auftraggeber bis Sonntag, 18 Uhr an Alexander.
\item[Alexander] Reserviert Zimmer f�r Kundensitzung und teilt dies dem Team mit.
\end{description}

\noindent\rule{\textwidth}{0.4pt}

\vfill

\begin{tabbing}
tab1 \= tab2 \= tab3 \= tab4 \= tab5 \= tab6 \= tab7 \= tab8 \= tab9 \= tab10 \= tab11 \= tab12 \= tab13 \= tab14 \kill

\>\>\> Besprechungsleiter \>\>\>\>\>\>\>\>\>\> Protokollf�hrer\\
\>\>\> Alexander Stocker \>\>\>\>\>\>\>\>\>\> Claudius J�rg\\
\>\>\> 04.03.2015 \>\>\>\>\>\>\>\>\>\> 04.03.2015\\
\end{tabbing}

\end{document}