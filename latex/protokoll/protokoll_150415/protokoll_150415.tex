\documentclass[a4paper]{article}
\usepackage[ngerman]{babel}
\usepackage[T1]{fontenc}
\usepackage[latin1]{inputenc}
\usepackage{lmodern}
\usepackage[top=2.5cm, bottom=2.5cm, left=2.5cm, right=2cm]{geometry}
\usepackage[final]{pdfpages}
\usepackage{graphicx}
\usepackage[colorlinks=true, urlcolor=blue]{hyperref}
\usepackage{multirow} 
\usepackage[headsepline,footsepline]{scrpage2} %[headtopline,headsepline,footsepline,footbotline]
\pagestyle{scrheadings}


\title{%
  \textsc{Projekt 2}\\[2ex]
  \textsc{Protokoll}}
\author{%
  \textsc{Team1}}
\date{%
  \textsc{}}

\clearscrheadfoot

%\tiny
%\scriptsize
%\footnotesize 
%\small 
%\normalsize 
%\large 
%\Large 
%\LARGE 
%\huge 
%\Huge 

\begin{document}
%Kopfzeile
\ihead{\includegraphics[scale=1]{fhnw_ht_logo_de}}
%\chead{Kopfzeile Mitte}
%Datum anpassen***************************************************************
\ohead{Protkoll vom 14.04.2015} 

%Fusszeile
\ifoot{Projekt 2}
\cfoot{Team 1}
\ofoot{\pagemark}

%Titel
%Datum anpassen**************************************************************
\section*{\textbf{Protokoll zur Sitzung am 15.04.2015}}

%F�r normale Schriftgr�sse
\normalsize
\begin{center}
\begin{tabular}{|l|l l|}
\hline
%Datum, Zeit, Ort anpassen*************
Datum, Zeit, Ort & \multicolumn{2}{|l|}{15.04.2015, 15:30-16:10 Uhr, FHNW Windisch, Geb�ude 5, Campus Bar}\\ \hline
Besprechungsleiter & Alexander Stocker & EIT Student / Projektleiter \\ \hline

%Anzahl Anwesende bestimmt die Zahl von multirow**********
\multirow{6}{*}{Teilnehmer} & Alexander Stocker & EIT Student / Projektleiter \\
& Claudius J�rg & EIT Student / Projektleiter Stellvertreter \\
& Martin Moser & EIT Student / Fachspezialist Technik \\
& Reto Freivogel & EIT Student / Fachspezialist Java \\
& Denis Stampfli & EIT Student / Java \\
& Yohannes Measho & EIT Student / Technik \\ \hline

%NICHT VERAENDERN*******************
\multirow{6}{*}{Verteiler} & Alexander Stocker & \href{mailto:alexander.stocker@students.fhnw.ch}{alexander.stocker@students.fhnw.ch} \\
& Claudius J�rg & \href{mailto:claudius.joerg@students.fhnw.ch}{claudius.joerg@students.fhnw.ch} \\
& Martin Moser & \href{mailto:martin.moser1@students.fhnw.ch}{martin.moser1@students.fhnw.ch} \\
& Reto Freivogel & \href{mailto:reto.freivogel@students.fhnw.ch}{reto.freivogel@students.fhnw.ch} \\
& Denis Stampfli & \href{mailto:denis.stampfli@students.fhnw.ch}{denis.stampfli@students.fhnw.ch} \\
& Yohannes Measho & \href{mailto:yohannes.measho@students.fhnw.ch}{yohannes.measho@students.fhnw.ch} \\ \hline

Sitzungsziele & \multicolumn{2}{|l|}{Siehe Traktandenliste}\\ \hline

%Anzahl Traktanden bestimmt die Zahl von multirow**********
\multirow{4}{*}{Traktandenliste} & \multicolumn{2}{|l|}{Begr�ssung}\\
& \multicolumn{2}{|l|}{Administratives}\\
& \multicolumn{2}{|l|}{Aktueller Stand}\\
& \multicolumn{2}{|l|}{Weiteres Vorgehen / Ausblick}\\ \hline

Protokollf�hrer & Claudius J�rg & EIT Student \\ \hline
\end{tabular}\newline
\end{center}
\noindent\rule{\textwidth}{0.4pt}
\subsection*{\textbf{Begr�ssung}}
\subsubsection*{\tiny{DISKUSSION}}

Der Sitzungsleiter begr�sst alle Teilnehmer und stellt kurz die Traktandenliste vor.

\noindent\rule{\textwidth}{0.4pt}
\subsection*{\textbf{Administratives}}%**********
\subsubsection*{\tiny{DISKUSSION}}

Stocker spricht Reto an bez�glich Erreichbarkeit.
\newline
Reto wird gebeten mindesten einmal pro Tag die Projektgruppe auf neue Nachrichten zu checken.

\subsubsection*{\tiny{BESCHLUSS}}


\noindent\rule{\textwidth}{0.4pt}
\subsection*{\textbf{Aktueller Stand}}%**********
\subsubsection*{\tiny{DISKUSSION}}

Moser hat Befehl Spline in Matlab erkundet. Der Befehl gibt den Funktionswert von f(x) aus einer Funktion mit 50 definierten
Punkten.
\newline
Reto hat die MathClass von Prof. Gut angeschaut.

\subsubsection*{\tiny{BESCHLUSS}}

Keine

\noindent\rule{\textwidth}{0.4pt}
\subsection*{\textbf{Weiteres Vorgehen / Ausblick}}
\subsubsection*{\tiny{DISKUSSION}}

Stocker will in drei Wochen einen grossen Teil (Details unten) von Matlab in Java �bersetzt haben.
\newline
Folgende Punkte sind in Java zu �bersetzen: Phasengangmethode aufrufen von Java in Matlab, das GUI mit Parameter f�r
Reglerstrecke und Ein-/Ausgabe, Faustformeln.
\newline
Die Phasengangmethode wird in einer Sitzung gemeinsam besprochen und aufgeteilt �bersetzt.

\subsubsection*{\tiny{BESCHLUSS}}

\begin{description}
\item[Alex]
\begin{itemize}
\item Erstellt Sitzungseinladung vom 15.04. und 22.04.15
\item Schreibt das GUI mit Parameter, Ein- und Ausgabe bis Sonntag, 19.04.15
\end{itemize}
\item[Claudius]
\begin{itemize}
\item Beteiligt sich an der �bersetzung der Faustformeln.
\item Erstellt Inhaltsverzeichnis des PMA-Berichts und hilft Yohannes in Latex
\end{itemize}
\item[Denis] Schreibt alle Faustformeln nach einer Verst�ndnissitzung mit Reto und Claudius bis Mittwoch, 22.04.15 (Struktur ist
bestehend auf GitHub)
\item[Martin]
\begin{itemize}
\item Schaut den Befehl Spline in der Bibliothek Math3 nach
\item Optimiert Matlabfunktionen
\item Bereitet Sitzung vor f�r alle. Thema: Phasengangmethode �bersetzen. Ziel: alleverstehen sie und k�nnen sie schreiben in
Java. Bis Mittwoch, 22.04.15
\end{itemize}
\item[Reto]
\begin{itemize}
\item Schreibt den Aufruf der Phasengangmethode in Java
\item Erstellt To-Do Katalog f�r Java
\item Wandelt Klassendiagramm in ein PNG um
\end{itemize}
\item[Yohannes] Erstellt Inhaltsverzeichnis f�r Fachbericht in Latex in Arbeit mit Claudius
\end{description}

\noindent\rule{\textwidth}{0.4pt}

\vfill

\begin{tabbing}
tab1 \= tab2 \= tab3 \= tab4 \= tab5 \= tab6 \= tab7 \= tab8 \= tab9 \= tab10 \= tab11 \= tab12 \= tab13 \= tab14 \kill

\>\>\> Besprechungsleiter \>\>\>\>\>\>\>\>\>\> Protokollf�hrer\\
\>\>\> Alexander Stocker \>\>\>\>\>\>\>\>\>\> Claudius J�rg\\
\>\>\> 15.04.2015 \>\>\>\>\>\>\>\>\>\> 15.04.2015\\
\end{tabbing}


%\begin{description}
%\item[Alex] Analyse Schrittantworten
%\item[Claudius] Rechercheordner auf GI erstellen, Thema: Streckenidentifikation
%\item[Denis] Thema: Amplitudengang
%\item[Martin] Analyse Schrittantworten, Thema: Phasengang
%\item[Reto] Layout Entwurf von GUI, Analyse Schrittantworten, Thema: Phasengangmethode/Frequenzsimulation
%\item[Yohannes] Thema: Mathematische �bertragungsfunktion
%\end{description}
\end{document}