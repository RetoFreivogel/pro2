\documentclass[a4paper]{article}
\usepackage[ngerman]{babel}
\usepackage[T1]{fontenc}
\usepackage[latin1]{inputenc}
\usepackage{lmodern}
\usepackage[top=2.5cm, bottom=2.5cm, left=2.5cm, right=2cm]{geometry}
\usepackage[final]{pdfpages}
\usepackage{graphicx}
\usepackage[colorlinks=true, urlcolor=blue]{hyperref}
\usepackage{multirow} 
\usepackage[headsepline,footsepline]{scrpage2} %[headtopline,headsepline,footsepline,footbotline]
\pagestyle{scrheadings}


\title{%
  \textsc{Projekt 2}\\[2ex]
  \textsc{Protokoll}}
\author{%
  \textsc{Team1}}
\date{%
  \textsc{}}

\clearscrheadfoot

%\tiny
%\scriptsize
%\footnotesize 
%\small 
%\normalsize 
%\large 
%\Large 
%\LARGE 
%\huge 
%\Huge 

\begin{document}
%Kopfzeile
\ihead{\includegraphics[scale=1]{fhnw_ht_logo_de}}
%\chead{Kopfzeile Mitte}
%Datum anpassen***************************************************************
\ohead{Protokoll vom 22.04.2015} 

%Fusszeile
\ifoot{Projekt 2}
\cfoot{Team 1}
\ofoot{\pagemark}

%Titel
%Datum anpassen**************************************************************
\section*{\textbf{Protokoll zur Sitzung am 22.04.2015}}

%F�r normale Schriftgr�sse
\normalsize
\begin{center}
\begin{tabular}{|l|l l|}
\hline
%Datum, Zeit, Ort anpassen*************
Datum, Zeit, Ort & \multicolumn{2}{|l|}{22.04.2015, 16:30-16:45 Uhr, FHNW Windisch, Geb�ude 1, 3. Etage}\\ \hline
Besprechungsleiter & Alexander Stocker & EIT Student / Projektleiter \\ \hline

%Anzahl Anwesende bestimmt die Zahl von multirow**********
\multirow{6}{*}{Teilnehmer} & Alexander Stocker & EIT Student / Projektleiter \\
& Claudius J�rg & EIT Student / Projektleiter Stellvertreter \\
& Martin Moser & EIT Student / Fachspezialist Technik \\
& Reto Freivogel & EIT Student / Fachspezialist Java \\
& Denis Stampfli & EIT Student / Java \\
& Yohannes Measho & EIT Student / Technik \\ \hline

%NICHT VERAENDERN*******************
\multirow{6}{*}{Verteiler} & Alexander Stocker & \href{mailto:alexander.stocker@students.fhnw.ch}{alexander.stocker@students.fhnw.ch} \\
& Claudius J�rg & \href{mailto:claudius.joerg@students.fhnw.ch}{claudius.joerg@students.fhnw.ch} \\
& Martin Moser & \href{mailto:martin.moser1@students.fhnw.ch}{martin.moser1@students.fhnw.ch} \\
& Reto Freivogel & \href{mailto:reto.freivogel@students.fhnw.ch}{reto.freivogel@students.fhnw.ch} \\
& Denis Stampfli & \href{mailto:denis.stampfli@students.fhnw.ch}{denis.stampfli@students.fhnw.ch} \\
& Yohannes Measho & \href{mailto:yohannes.measho@students.fhnw.ch}{yohannes.measho@students.fhnw.ch} \\ \hline

Sitzungsziele & \multicolumn{2}{|l|}{Siehe Traktandenliste}\\ \hline

%Anzahl Traktanden bestimmt die Zahl von multirow**********
\multirow{4}{*}{Traktandenliste} & \multicolumn{2}{|l|}{Begr�ssung}\\
& \multicolumn{2}{|l|}{Administratives}\\
& \multicolumn{2}{|l|}{Aktueller Stand}\\
& \multicolumn{2}{|l|}{Weiteres Vorgehen / Ausblick}\\ \hline

Protokollf�hrer & Claudius J�rg & EIT Student \\ \hline
\end{tabular}\newline
\end{center}
\noindent\rule{\textwidth}{0.4pt}
\subsection*{\textbf{Begr�ssung}}
\subsubsection*{\tiny{DISKUSSION}}

Der Sitzungsleiter begr�sst alle Teilnehmer und stellt kurz die Traktandenliste vor.

\noindent\rule{\textwidth}{0.4pt}
\subsection*{\textbf{Administratives}}%**********
\subsubsection*{\tiny{DISKUSSION}}

Teammitglieder arbeiten f�r ihre Berichte nur in den Unterkapitel-Dateien von Latex (title\_subtitle.tex)
\newline
Die Monte-Carlo-Analyse bleibt pendent.

\subsubsection*{\tiny{BESCHLUSS}}



\noindent\rule{\textwidth}{0.4pt}
\subsection*{\textbf{Aktueller Stand}}%**********
\subsubsection*{\tiny{DISKUSSION}}

Alex hat Layout erstellt und optimiert mit Prof. R. Gut.
\newline
Martin hat sich mit Matlab Befehl SPLINE vertraut gemacht und gibt Informationen per Mail an Programmierer weiter.
\newline
Reto arbeitet nach wie vor am Klassendiagramm.

\subsubsection*{\tiny{BESCHLUSS}}

\begin{description}
\item[Martin] Alle erhalten die Informationen von SPLINE zum programmieren per Mail. Bis Freitag, 24.04.2015.
\item[Reto] Klassendiagramm erstellen. Fortschritt bei n�chster Sitzung zeigen.
\end{description}

\noindent\rule{\textwidth}{0.4pt}
\subsection*{\textbf{Weiteres Vorgehen / Ausblick}}
\subsubsection*{\tiny{DISKUSSION}}

Disposition mit �bersicht des Inhaltes an A. Gertiser bis 01. Mai.
\newline
Bis jetzt kein Feedback Pflichtenheft von Prof. R. Gut.

\subsubsection*{\tiny{BESCHLUSS}}

\begin{description}
\item[Alex]
\begin{itemize}
\item Disposition schreiben und A. Gertiser senden bis 01.05.2015.
\item  F�r Pflichtenheft bei Niklaus nachfragen
\end{itemize}
\item[Claudius] Schreibt alle Protokolle in Latex
\item[Denis] 
\begin{itemize}
\item Faustformel Chiens/Hoswick/Reswick programmieren und testen. Bis 29.04.2015.
\item  sani-File umschreiben (Stichwort: Befehl SPLINE). Informationen von Martin.
\end{itemize}
\item[Martin] ?Erkl�rung Phasengangmethode? verschoben auf Freitag, 24.04.2015, 11 Uhr (Wenn alle fertig mir unterricht)
\item[Reto]
\begin{itemize}
\item Aufwandssch�tzung f�r Phasengangmethode programmieren: TODO-Liste f�r Alex. Erste Edition in einer Woche.
\item  Befasst sich mit dem programmieren der Zellweger-Methode
\end{itemize}
\item[Yohannes] steht zur Verf�gung Team
\end{description}

\noindent\rule{\textwidth}{0.4pt}

\vfill

\begin{tabbing}
tab1 \= tab2 \= tab3 \= tab4 \= tab5 \= tab6 \= tab7 \= tab8 \= tab9 \= tab10 \= tab11 \= tab12 \= tab13 \= tab14 \kill

\>\>\> Besprechungsleiter \>\>\>\>\>\>\>\>\>\> Protokollf�hrer\\
\>\>\> Alexander Stocker \>\>\>\>\>\>\>\>\>\> Claudius J�rg\\
\>\>\> 22.04.2015 \>\>\>\>\>\>\>\>\>\> 22.04.2015\\
\end{tabbing}


%\begin{description}
%\item[Alex] Analyse Schrittantworten
%\item[Claudius] Rechercheordner auf GI erstellen, Thema: Streckenidentifikation
%\item[Denis] Thema: Amplitudengang
%\item[Martin] Analyse Schrittantworten, Thema: Phasengang
%\item[Reto] Layout Entwurf von GUI, Analyse Schrittantworten, Thema: Phasengangmethode/Frequenzsimulation
%\item[Yohannes] Thema: Mathematische �bertragungsfunktion
%\end{description}
\end{document}