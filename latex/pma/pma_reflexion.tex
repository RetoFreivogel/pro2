\subsection{Positive Erkenntnisse}
\textbf{Arbeit}\\
Die wöchentliche Zusammenarbeit förderte den Teamgeist. Das Team nutzte die gemeinsame Zeit, um sich auszutauschen und kreativ zu wirken. Trotz unübersichtlichen Sitzungen waren die Aufgaben für jedes Teammitglied klar. Im ersten Teil des Projektes waren die Aufgaben eher auf den elektrotechnischen Teil ausgelagert. In der zweiten Phase verlagerte sich das Schwergewicht zur die Java-Gruppe. Das Team bestand aus sehr kompetenten Fachkräften, welche sich nötiges Wissen selbständig aneigneten. Sie suchten für jedes Problem eine Lösungen oder stellten konkrete Fragen jeweils direkt dem Fachcoach. Die Motivation war während der ganzen Projektphase relativ hoch, obwohl das Team anfangs Schwierigkeiten mit der Materie hatte.\\ 
Die Regelungstechnik hat enorm viele Fachbegriffe, die das Verständnis der Aufgabe erschwerten. Deshalb beschäftigte sich die Fachgruppe zuerst mit der Klärung der Begriffe, sodass die Aufgabenstellung verständlicher wurde. Anschliessend setzten sie sich mit den vorgegebenen Matlab-Skripten auseinander. Dadurch konnte sich die Fachgruppe einen Ablauf zusammenstellen, welcher später für die Planung der Software wichtig war. Die Softwaregruppe konnte sich zu dieser Zeit mit der graphischen Oberfläche auseinandersetzten. Sie entwarfen zwei Entwürfe und stellten diese dem Kunden vor. Durch die frühe Rückmeldung des Kunden, konnten die graphische Oberfläche nach einigen Anpassung verwendet werden.
Trotz des Engagements des Teams kam es dennoch vor, dass Termine nicht eingehalten wurden. Was zu kleinen Verzögerungen führte. Durch die eingeplante Reserve hatten die Verzögerungen keine Relevanz.\\ 

\textbf{Organisation}\\
Die Aufteilung des Projektteams in kleine Gruppen war ein entscheidender Faktor. Die Gruppen konnte zielorientiert arbeiten und deckten zeitgleich verschiedene Themen ab. Die Zuteilung der Aufgaben konnten somit einfacher und zielgerichteter formuliert werden. Folglich war die Kontrolle für den Projektleiter einfacher und die folgenden Phasen wurden besser geplant.\\ 
Der Anfang war jedoch für den Projektleiter im Aufgabenbereich Projektmanagement schwierig. Das Team tat sich schwer mit der Auftragsklärung. Trotz vertieften Recherchen schwebten Unklarheiten umher. Unglücklicherweise blockierten die Unklarheiten die Planung der Arbeitspakete. Da es ohne den Durchblick unmöglich war eine Planung der Arbeitspakete zu machen, ging das Team den Unklarheiten auf den Grund. Durch verschiedene Input Referate wurde das Wissen über Regelungstechnik im Team erweitert. Demnach war es trotz der kompakt gestellten Aufgabenstellung schwer, Licht in die Dunkelheit zubringen und die Unklarheiten zu klären. Die Gruppe Elektrotechnik erarbeitete sich aus den Inputs und den Skripten ein Flussdiagramm. Dieses zeigt die verschiedenen Wege der Dimensionierung auf. Klar im Fokus steht die Dimensionierung mit der Phasengangmethode. Die Möglichkeiten mit Faustformeln sind im Diagramm ebenfalls ersichtlich. Das Diagramm nutzte der Verantwortliche der Elektrotechnikgruppe, um dem Team den Ablauf zu erklären. Zudem wurde das Diagramm zur Planung der Software benötigt. Auch der Projektleiter konnte seinen Nutzen daraus ziehen. Dank dem Diagramm war der Aufwand fassbarer. Eine minimale Aufwandschätzung entstand daraus und Arbeitspakete wurden davon abgeleitet. Nach dem der Arbeitsplan erstellt wurde, kristallisierte sich ein weiteres organisatorisches Problem heraus. Der Software-Spezialist nannte den Ablauf der Programmierphase nur teilweise sinnvoll. Grundlegend fand er die Planung der Arbeitspakete gut, doch er wollte unbedingt mit der graphische Darstellung der Schrittantwort beginnen. Darauf folgte eine ausserordentliche Sitzung zwischen Projektleiter und dem Java-Spezialisten. Das Resultat war einen klaren Ablauf und eine Umstrukturierung des Plans. Zu dem steigerte diese Sitzung das Arbeitsklima im Team.\\ 
Die Planung und Organisation der Aufgabe hat ein einen hohen Stellenwert in Projekten. Besonders wichtig wird es, wenn ein Projektteam aus mehreren Personen besteht. Zu dem ist es sinnvoll, wenn das Team aufgeteilt wird. So hat es jeweils Gruppenverantwortliche, welche im engem Kontakt mit dem Projektleiter stehen. Abschliessend kristallisierte sich die Organisation als ein positives Ereignis aus.\\ 
Zukünftig wird der Projektleiter die Wünsche seiner Teammitglieder berücksichtigen und in die Planung einfliessen lassen. Sobald ein Plan steht sollte dieser mit dem Team besprochen und sinngemäss angepasst werden. Zu dem sind regelmäßige Sitzungen, in welchen der Fortschritt besprochen wird, von Vorteil.\\

\subsection{Negative Erkenntnisse}
\textbf{Sitzungen}\\
Sitzungen sind das Führungsmittel des Projektleiters. Der Austausch von Informationen und erarbeitete Lösungen, wie auch die Verteilungen neuer Aufgaben wurden an Sitzungen thematisiert. Der Projektleiter hatte anfangs ein Standard eingeführt, dass Ort, Zeit und Ablauf immer gleich bleibt. Vor jeder Sitzung sendete der Projektleiter die Einladungen mit Traktanden. Trotz den ausführlichen Einladungen, war die Vorbereitung sowohl vom Projektleiter als auch von den Teammitglieder ungenügend. Die Sitzungen wirkten lahm. Zusätzlich kam das Team permanent vom Thema ab und verlor sich in den Details. Folglich wurde es für den Protokollführer schwierig sich an die Struktur halten. Vielfach musste er zwischen Ausblick und Rückblick hin und her springen, um alle Stichworte zu erfassen. Doch dies war zu anspruchsvoll, sodass die Protokollen die gleiche Struktur wie die Sitzung hatte. Die Sitzungsprotokolle war folglich unbrauchbar. Als weiterer Aspekt dauerten die Sitzungen  zu lange und der Informationsaustausch wirkte auf erzwungen. Ganz anders als die zwischen Gespräche während dem Arbeiten. Diese brachten meist mehr Informationen als die Sitzungen selbst. Dem Projektleiter war diese Situation bekannt und er versuchte Einfluss zunehmen. Mit seinem ersten Versuch die Struktur der Einladung einzuhalten brachte er es dennoch nicht hin wichtige Informationen zu gewinnen und diese für weitere Aufgaben weiter zu verwenden. Er beschloss also die Sitzungen zu verkürzen und nur das nötigste zu diskutieren. So konnte er vermeiden, dass einzelne gedanklich ausklinken. 
Als Erkenntnisse bleibt schlecht vorbereitete Sitzungen machen die Führung schwieriger. Für zukünftige Projekte werden Sitzungen besser geplant und strikt nach Plan geführt. So besteht die Gefahr nicht, dass jemand Informationslücken hat. Zudem wird zukünftig das Protokoll genauer auf Lücken geprüft.\\

\textbf{Auftragsklärung}\\
Dem Team war sich zu Beginn des Projektes nicht bewusst, dass die Aufgabenstellung von Herrn Niklaus sehr kompakt geschrieben wurde. Obwohl alle die Aufgabenstellung durchlasen, wurde markante Punkte nicht berücksichtig. Die Recherchen wurden folglich ungenügend, was wiederum zu einer schlechte Zielformulierung führte. Dies führte wiederum zu einer Zeiteinbusse, welche vom Projektleiter als negative Erkenntnis gedeutet wurde.\\ 
Anfangs war der Wissensstand vom Team über Regelungstechnik begrenzt. Einige Teammitglieder waren auf Grund ihrer Vorkenntnisse besser darüber informiert als die anderen. In der ersten Sitzung wurde die Aufgabenstellung vom Team analysiert und Recherchethemen formuliert. Der Projektleiter teilte jedem Mitglied ein Thema zu. Das Team hatte eine Woche Zeit, um zu recherchieren und die Themen zusammenzufassen. Die Arbeit stellte sich als ungenügend heraus. Aufgrund der hohen Anforderungen der Mathematik, dem Begriffschaos der Regelungstechnik und dem schwerverständlichen Zellwegerskript hatte das Team Schwierigkeiten die Ziele zu formulieren. Vom Fachcoach standen zwar Matlab-Skripte zur Verfügung. Diese brachten vorerst aber keinen Erfolg. In der dritten Woche wurde auf Grund der Ereignisse beschlossen, dass das Elektrotechnikteam die Verantwortung über die Recherchearbeit übernimmt. Ziel war es eine Übersicht zu schaffen und spezifischer zu recherchieren. Der Elektrotechnikspezialist untersuchte vorerst die Skripte über Regelungstechnik und konnte gewinnbringende Zusammenhänge herausfiltern. Trotz den Resultaten der Recherche waren weiterhin die nötigen Zusammenhänge nicht vorhanden. Es fanden wöchentlich Fachinputs statt, in welchen die nötige Theorie der Regelungstechnik erläutert wurde. Zu dem konnten die Teams Fragen über Formeln und Zusammenhänge stellen. Diese Inputs brachten dem Elektrospezialisten des Teams einen klaren Durchblick. Die Elektrotechnikgruppe konnte die Zusammenhänge graphisch, in Form eines Ablaufdiagramms, darstellen.\\
Wegen der späten Erkenntnis konnte im Softwarebereich schlecht recherchiert werden. Das Team wusste zu Beginn des Projektes nicht, wie komplex die Software wurde. Zu dem baute die Recherche der Software Gruppe auf der Recherche der Elektrotechnikgruppe auf. 
Rückblickend wurde die Phase der Analyse zu schlecht genutzt. Mit einer genaueren Analyse der Aufgabenstellung und konkreteren Recherchethemen kann besser recherchiert werden. Ein Glossar mit fachspezifischen Stichworten hilft den Überblick zu behalten. Regelmässigere Sitzungen mit dem Auftraggeber helfen dem Team, die Absichten und Wünsche des Auftraggebers zu verdeutlichen.
Zukünftig wird sich das Team Zeit nehmen, um die Aufgabenstellung auseinander zunehmen. Diese wird diskutiert und Entscheide werden festgehalten. So kann zukünftig Zeit eingespart werden und die Planung präziser formuliert werden.

