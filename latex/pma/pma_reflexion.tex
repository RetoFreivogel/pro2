\subsection{Arbeit}
Die wöchentliche Zusammenarbeit förderte den Teamgeist. Das Team nutzte die gemeinsame Zeit , um sich auszutauschen und kreativ zu wirken. Trotz unübersichtlichen Sitzungen waren die Aufgaben für jedes Teammitglied klar. Im ersten Teil des Projektes waren die Aufgaben eher auf dem elektrotechnischen Teil ausgelagert. In der zweiten Phase verlagerte sich das Schwergewicht zur die Java-Gruppe. Das Team bestand aus sehr kompetenten Fachkräften, welche sich nötiges Wissen selbständig aneigneten. Sie suchten für jedes Problem eine Lösungen oder stellten konkrete Fragen jeweils direkt dem Fachcoach. Die Motivation war während der ganzen Projektphase relativ hoch, obwohl das Team anfangs Schwierigkeiten mit der Materie hatte. 
Die Regelungstechnik hat enorm viele Fachbegriffe, die das Verständnis der Aufgabe erschwerten. Deshalb beschäftigte sich die Fachgruppe zuerst mit der Klärung der Begriffe, sodass die Aufgabenstellung verständlicher wurde. Anschliessend setzten sie sich mit den vorgegebenen Matlab-Skripten auseinander. Dadurch konnte sich die Fachgruppe einen Ablauf zusammen stellen, welcher später für die Planung der Software wichtig war. Die Softwaregruppe konnte sich zu dieser Zeit mit der graphischen Oberfläche auseinandersetzten. Sie entwarfen zwei Entwürfe und stellten diese dem Kunden vor. Durch die frühe Rückmeldung des Kunden, konnten die graphische Oberfläche nach einigen Anpassung  verwendet werden.
Trotz des Engagement des Teams kam es dennoch vor, dass Termine nicht eingehalten wurden. Was zu kleinen Verzögerungen führte. Durch die eingeplante Reserve hatten die Verzögerungen keine Relevanz. 

 
\subsection{Sitzung}
Sitzungen sind das Führungsmittel des Projektleiters. Der Austausch von Informationen und erarbeitete Lösungen, wie auch die Verteilungen neuer Aufgaben wurden an Sitzungen thematisiert. Der Projektleiter hatte anfangs ein Standard eingeführt, dass Ort, Zeit und Ablauf immer gleich bleibt. Doch die Erfahrung zeigte, dass der Standard nicht umsetzbar war. Die Vorbereitung auf die Sitzungen waren vom Team ungenügend erfüllt worden. Die Denn Wichtig für eine Sitzung ist schliesslich die Vorbereitung der Teilnehmer. Mit der Vorbereitung kann eine Sitzung effizient durchgeführt werden. Das widerspiegelt sich im Protokoll wieder. Was zum Anfang sehr gut klappte und mit wie viel Elan die Arbeiten angegangen wurden, flaute die Sorgfaltspflicht mit dem Fortschreiten des Projekts allmählich ab. Die Sitzungen wurden immer kürzer, weil nicht mehr viel Fachliches diskutiert wurde. Die Fachgruppen arbeiteten ziemlich autonom und so entwickelte sich die Sitzung zum Verteiler von Arbeiten und zur Registrierung des Arbeitspensums. Die ersten Sitzungen waren lang und das Ganze Team diskutierte technische Lösungen. So kam es, dass einzelne Mitglieder nicht mehr aufmerksam zuhörten, weil sie nicht zur Fachgruppe Elektrotechnik gehörten beispielsweise. Das ist die Folge der schlechten Vorbereitung der Teilnehmer, wenn sie bei ihrer Aufgabe auf Probleme gestossen sind und ihr Anliegen für eine saubere Diskussion in der Sitzung nicht aufbereitet haben. Darum wurden die Sitzungen verkürzt und es wurden nur noch kurze Lösungsansätze behandelt. Technische Probleme diskutierten die Fachgruppen bilateral, das führte zu einem unterschiedlichen Wissensstand im Team. Das führte dazu, dass der Austausch in Sitzungen fehlte und das allgemeine Verständnis im Projekt sank. Ein weiterer Punkt waren die Sitzungsstrukturen. Obwohl auf den Sitzungseinladungen die Diskussionspunkte aufgeteilt waren, wurde der Ablauf nur vage eingehalten. Das führte zu unübersichtlichen Sitzungen und hatte zur Folge, dass die nicht involvierten Personen den roten Faden verloren. Das bedingt, dass der Sitzungsleiter sich an diesen Ablauf hält und die Unbeteiligten die Themen später nachlesen können im Protokoll.
Die Erkenntnisse aus den Sitzungen sind bessere Vorbereitung jedes einzelnen, um die Probleme klar aufzuzeigen und effizient zu diskutieren im Team, ein klarer Aufbau und Durchführung der Sitzung durch den Sitzungsleiter.
\subsection{Organisation}
Die Aufteilung des Projektteams in kleine Gruppen war ein entscheidender Faktor. Die Gruppen konnte Zielorientiert arbeiten und deckte zeitgleich verschiedene Themen ab. Die Zuteilung der Aufgaben konnten somit einfacher und zielgerichteter Formuliert werden. Folglich war die Kontrolle für den Projektleiter einfacher und die folgenden Phasen wurden besser geplant. 
Der Anfang war jedoch für den Projektleiter im Aufgabenbereich Projektmanagement schwierig. Das Team tat sich schwer mit der Analyse der Aufgabe. Trotz vertieften Recherchen schwebten Unklarheiten umher. Unglücklicherweise blockierten die Unklarheiten die Planung der Arbeitspakete. Da es ohne den Durchblick unmöglich war eine Planung der Arbeitspakete zu machen, ging das Team den Unklarheiten auf den Grund. Durch verschiedene Input Referate wurde das Wissen  über Regelungstechnik im Team erweitert. Demnach war es trotz der kompakt gestellten Aufgabenstellung licht in die Dunkelheit zubringen und die Unklarheiten zu klären. Die Gruppe Elektrotechnik erarbeitete sich aus den Inputs und den Skripten ein Flussdiagramm. Dieses zeig die verscheiden Wege der Dimensionierung auf. Klar im Fokus steht die Dimensionierung mit der Phasengangmethode. Die Möglichkeiten mit Faustformeln sind im Diagramm ebenfalls ersichtlich. Das Diagramm nutzte der verantwortliche der Elektrotechnikgruppe, um dem Team den Ablauf zu erklären. Zudem wurde das Diagramm zur Planung der Software benötigt. Auch der Projektleiter konnte seinen Nutzen daraus ziehen. Dank dem Diagramm war der Aufwand fassbarer. Eine minimale Aufwandschätzung entstand daraus und Arbeitspakete wurden davon abgeleitet. Nach dem der Arbeitsplan erstellt wurde, kristallisierte sich ein weiteres organisatorisches Problem heraus. Der Software-Spezialist nannte den Ablauf der Programmierphase nur teilweise sinnvoll. Grundlegend fand er die Planung der Arbeitspakete gut, doch er wollte unbedingt mit der graphische Darstellung der Schrittantwort beginnen. Darauf folgte eine ausserordentliche Sitzung zwischen Projektleiter und dem Java-Spezialisten. Das Resultat war einen klaren Ablauf und um Strukturierung des Plans. Zu dem steigerte diese Sitzung das Arbeitsklima im Team. 
Die Planung und Organisation der Aufgabe hat ein einen hohen Stellenwert in Projekten. Besonders wichtig wird es, wenn ein Projektteam aus mehreren Personen besteht. Zu dem ist es Sinnvoll, wenn das Team aufgeteilt wird. So hat es jeweils Gruppenverantwortlichen, welche im engen Kontakt mit dem Projektleiter stehen. Abschliessend kristallisierte sich die Organisation als ein positives Ereignis aus. 
Zukünftig wird der Projektleiter die Wünsche seiner Teammitglieder berücksichtigen und in die Planung einfliessen lassen. Sobald ein Plan steht sollte dieser mit dem Team besprochen und sinngemäss angepasst werden. Zu dem sind regelmäßige Sitzungen, in welchen der Fortschritt besprochen wird, von Vorteil.
\subsection{Auftragsklärung}
Dem Team war es zu Begin des Projektes nicht bewusst, dass die Aufgabenstellung von Herrn Niklaus sehr kompakt geschrieben wurde. Obwohl alle die Aufgabenstellung durchlasen, wurde markante Punkte nicht berücksichtig. Die Recherchen wurden folglich ungenügend, was wiederum zu einer schlechte Zielformulierung führte. Dies führte wiederum zu einer Zeiteinbusse, welche vom Projektleiter als negative Erkenntnis gedeutet wurde. 
Anfangs war der Wissensstand vom Team über Regelungstechnik begrenzt. Einige Teammitglieder waren auf Grund ihrer Vorkenntnisse besser darüber informiert als die Anderen. In der ersten Sitzung wurde die Aufgabenstellung vom Team analysiert und Recherchethemen formuliert. Der Projektleiter teilte jedem Mitglied ein Thema zu. Das Team hatte eine Woche Zeit, um zu recherchieren und die Themen zusammen zufassen. Die Arbeit stellte sich als ungenügend heraus. Die Teammitglieder sahen die Zusammenhänge nicht und konnten keine Schlüsse ziehen. Ob es die hohe Anforderungen der Mathematik, das Begriffschaos der Regelungstechnik oder das schwerverständliche Zellwegerskript war, das Team hatte Schwierigkeiten die Ziele zu formulieren. Vom Fachcoach standen zwar Matlab-Skripte zur Verfügung. Diese brachten vorerst keinen Erfolg. In der dritten Woche wurde auf Grund der Ereignisse beschlossen, dass das Elektrotechnikteam die Verantwortung über die Recherchearbeit übernimmt. Ziel war es eine Übersicht zu schaffen und spezifischer zu recherchieren. Der Elektrotechnikspezialist untersuchte vorerst die Skripte über Regelungstechnik und konnte gewinnbringende Zusammenhänge herausfiltern. Trotz den Resultaten der Recherche waren weiterhin die nötigen Zusammenhänge nicht vorhanden. Es fanden wöchentlich Fachinputs statt, in welchen die Aufgabenstellung erläutert wurde. Zu dem konnten die Teams Fragen über Formeln und Zusammenhänge stellen. Diese Inputs brachten dem Elektrospezialisten des Teams eins einen klaren Durchblick. Die Elektrotechnikgruppe konnte die zusammenhänge graphisch, in Form eines Ablaufdiagramms, darstellen. Wegen der spaten Erkenntnis konnte im Softwarebereich schlecht recherchiert werden. Das Team wusste zu beginn des Projektes nicht, wie komplex die Software wurde. Zu dem baute die Recherche der Software Gruppe auf der Recherche der Elektrotechnikgruppe auf. 
Rückblicken wurde die Phase der Analyse schlecht genutzt. Mit einer genaueren Analyse der Aufgabenstellung und konkreteren Recherchethemen kann besser Recherchiert werden. Ein Glossar mit Fachspezifischen Stichworten hilft den Überblick zu behalten. Regelmässigere Sitzungen mit dem Auftraggeber helfen dem Team, die Absichten und Wünsche des Auftraggebers zu verdeutlichen.
Zukünftig wird sich das Team Zeit nehmen, um die Aufgabenstellung auseinander zunehmen. Diese werden diskutiert und Entscheide werden festgehalten. So kann zukünftig Zeit eingespart werden und die Planung präziser formuliert werden.