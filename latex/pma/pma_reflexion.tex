\subsection{Arbeit}
Durch die klare Struktur im Team und die wöchentliche Besprechung, Diskussion und Zuweisung der Arbeit war jedem Teammitglied klar, welche Arbeit er zu erledigen hatte. So hatte jeder Mittwoch den gleichen Ablauf, dem immer eine Sitzungseinladung des Projektleiters voran ging. Der Nachmittag begann jeweils mit einer Sitzung zur Standortbestimmung, bei der jedes Teammitglied seine Aufgabe seit der letzten Sitzung kurz erläuterte und welche Probleme ihm dabei begegnet sind und ob die Arbeit abgeschlossen ist oder noch etwas Zeit benötigt, dann nahm der Projektleiter das Arbeitspensum auf. Mit der neuen Vergabe von Arbeiten konnte dann gleich begonnen werden und trotz gleitender Arbeitszeiten arbeiteten alle am gleichen Arbeitsplatz. Das gemeinsame Arbeiten hatte Nachteile, aber noch mehr Vorteile. Es motivierte im Team zu arbeiten, man erlebte das Arbeiten gemeinsam, tauschte sich aus und konnte einander direkt helfen. Der Nachteil lag im Ablenken, aber das war sehr selten. Es kam vor, dass an Arbeitsintensiven Nachmittagen eine Statussitzung einberufen wurde um Abweichungen zu korrigieren und den Fortschritt und das Ziel im Auge zu behalten. Weil die Arbeiten immer Wochenweise vergeben wurden war ein ausgeprägtes Pflichtbewusstsein wichtig. Denn es kam vor, dass in der folgenden Woche ein anderes Teammitglied am Arbeitspaket weiterarbeiten musste und bei unerledigten Arbeiten lief das Team Gefahr mit dem Projekt in Verzug zu kommen. Aus dem Pflichtbewusstsein resultierten eine saubere Arbeitsweise, ein verzugsfreies Projekt und ein stressfreies Arbeitsumfeld. Die Arbeit war somit eine positive Erfahrung, weil die Teammitglieder die Vorgaben und Arbeiten sauber umsetzten und für den Projektleiter übersichtlich in den Ordner und Dateien ablegten. Hierbei spielten die technischen Hilfsmittel eine zentrale Rolle. Mit GitHub zum Beispiel konnten am Programm mehrere Mitglieder arbeiten ohne, dass sie einander in den Weg kamen. Mit einer Synchronisierung mit dem Konto hatten dann alle die neuste Version zur Verfügung. Zusätzlich hatte der Projektleiter jederzeit die Übersicht über den Fortschritt von jedem einzelnen.
\subsection{Sitzung}
Sitzungen sind ein Führungsmittel des Projektleiters und ein Austausch von gemeinsamen Lösungen im Team. Der Projektleiter hat es geschafft ein Standard einzuführen, dass Ort, Zeit und Ablauf immer sehr ähnlich waren. Wichtig für eine Sitzung ist schliesslich die Vorbereitung der Teilnehmer. Mit der Vorbereitung kann eine Sitzung effizient durchgeführt werden. Das widerspiegelt sich im Protokoll wieder. Was zum Anfang sehr gut klappte und mit wie viel Elan die Arbeiten angegangen wurden, flaute die Sorgfaltspflicht mit dem Fortschreiten des Projekts allmählich ab. Die Sitzungen wurden immer kürzer, weil nicht mehr viel Fachliches diskutiert wurde. Die Fachgruppen arbeiteten ziemlich autonom und so entwickelte sich die Sitzung zum Verteiler von Arbeiten und zur Registrierung des Arbeitspensums. Die ersten Sitzungen waren lang und das Ganze Team diskutierte technische Lösungen. So kam es, dass einzelne Mitglieder nicht mehr aufmerksam zuhörten, weil sie nicht zur Fachgruppe Elektrotechnik gehörten beispielsweise. Das ist die Folge der schlechten Vorbereitung der Teilnehmer, wenn sie bei ihrer Aufgabe auf Probleme gestossen sind und ihr Anliegen für eine saubere Diskussion in der Sitzung nicht aufbereitet haben. Darum wurden die Sitzungen verkürzt und es wurden nur noch kurze Lösungsansätze behandelt. Technische Probleme diskutierten die Fachgruppen bilateral, das führte zu einem unterschiedlichen Wissensstand im Team. Das führte dazu, dass der Austausch in Sitzungen fehlte und das allgemeine Verständnis im Projekt sank. Ein weiterer Punkt waren die Sitzungsstrukturen. Obwohl auf den Sitzungseinladungen die Diskussionspunkte aufgeteilt waren, wurde der Ablauf nur vage eingehalten. Das führte zu unübersichtlichen Sitzungen und hatte zur Folge, dass die nicht involvierten Personen den roten Faden verloren. Das bedingt, dass der Sitzungsleiter sich an diesen Ablauf hält und die Unbeteiligten die Themen später nachlesen können im Protokoll.
Die Erkenntnisse aus den Sitzungen sind bessere Vorbereitung jedes einzelnen, um die Probleme klar aufzuzeigen und effizient zu diskutieren im Team, ein klarer Aufbau und Durchführung der Sitzung durch den Sitzungsleiter.