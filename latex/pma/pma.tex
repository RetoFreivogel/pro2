\documentclass{fhnwreport}
\usepackage[ngerman]{babel}
\usepackage[T1]{fontenc}
\usepackage[latin1]{inputenc}
\usepackage{tikz}
\usepackage{amsmath}
\usetikzlibrary{arrows}
\usepackage{lmodern}
\usepackage[final]{pdfpages}
\usepackage{graphicx}

\title{%
  \textsc{Projekt 2}\\[2ex]
  \textsc{Post-Mortum-Analyse}}
\author{%
  \textsc{Team1}}
\date{%
  \textsc{10.06.2015}}

%\textit{} % italics
%\textbf{} % bold
%\texttt{} % typewriter style
%\textsf{} % sans-serif
%\textsc{} % all capital letters

\begin{document}
\maketitle

\vfill

\textsc{%
\begin{tabbing}
Auftraggeber: \hspace{4em} \=  Peter Niklaus \\[2ex]
Betreuer:  \>  Pascal Buchschacher, Anita Gertiser \\[2ex]
Experten:  \>  Peter Niklaus, Richard Gut \\[2ex]
Team:  \> Alexander Stocker \\ 
\> Claudius J�rg \\
\> Denis Stampfli \\
\> Martin Moser \\
\> Reto Freivogel \\
\> Yohannes Measho \\ [2ex]
Studiengang: \> Elektro- und Informationstechnik
\end{tabbing}}

\clearpage

\tableofcontents
\newpage

\section{Einleitung}
%Ausgangslage: Worum ging es im Projekt (sehr kurz und b�ndig)
%Anforderungen: Wichtigste Projektziele, Bedingungen, die ber�cksichtigt werden mussten
%Organisation: Auf PM Ebene
%Resultate: Wichtigste positive und negative Erfahrungen
%Aufbau des Berichtes
\section{Analyse der Erfahrung}
%Umfang ca. 1-2 Seite
%Was waren die 6 wichtigsten Erfahrungen des Projekt-Teams (3 positive, 3 negative)?
%Nennen Sie stichwortartig die positiven bzw. negativen Erfahrungen und dessen Ursachen (Einflussfaktoren)
%in einem Ishikawa (fishbone) Diagramm
\section{Reflexion}
%Umfang ca. 4 Seiten
%Reflektieren Sie f�r jede Erfahrung (2 positive und 2 negative) auf ca. je einer Seite folgende Aspekte:
%Ausgangslag: Worum ging es bei der Erfahrung? Welches Risiko ist / welche Risiken sind (nicht) eingetreten? z.B. der reelle Aufwand f�r die AP XY war viel gr�sser / kleiner als gesch�tzt (sehr kurz und b�ndig)
%Abweichungen: Beschreiben Sie die jeweiligen Abweichungen (positiv oder negativ). Was hat dazu gef�hrt, das alles wie geplant funktionierte ? oder eben nicht? Welche Folgen zeigten sich daraus f�r das Projekt oder die Zusammenarbeit im Team?
%Massnahmen: Welche Massnahmen wurden ergriffen (oder h�tten ergriffen werden m�ssen), um die Abweichungen zu korrigieren? Wie wurden die Risiken bew�ltigt, oder was hat zur positiven Entwicklung gef�hrt?
%Effekt: Welche Wirkung(en) haben die ergriffenen Massnahmen (nicht) gezeigt?
%Lessons-learned: Welche Erkenntnisse k�nnen f�r weitere Projekte daraus gewonnen werden?
\section{Schlusswort}
%Umfang ca. � Seite
\section{Literaturverzeichnis}

\end{document}

