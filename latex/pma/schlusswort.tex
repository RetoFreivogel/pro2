Der Fortschritt im letzten Projekt war riesig. Ob schulische Projekte oder Firmenprojekte, das Team konnte wertvolle Erfahrungen für die Zukunft sammeln. Das Team arbeitete während des ganzen Projektes in einem angenehmen Arbeitsklima. Einerseits war dies möglich, weil jedes Mitglied von den anderen respektiert wurde. Andererseits kam von jedem Einzelnen eine objektive Äusserung, was sich als gewinnbringend fürs Team herausstellt. Es kam selten vor, dass die Motivation tief war. Ob der Projektleiter oder jemand anderes im Team der Auslöser dafür war ist unklar, doch solange das Team motiviert ist, ist der Erfolg garantiert.\\ 
Während der Projektwoche hatte das Team gemeinsam die Software geschrieben. Das Team konnte in dieser Woche ungestört und effizient arbeiten. Unglücklicherweise war das Team in den anderen Wochen nicht gleich zügig. Grund dafür war die zusätzliche Belastung der anderen Module. Doch falls sich jeder Einzelne Mühe gibt, könnte die Zügigkeit auch in den normal Unterrichtswochen aufrecht erhalten werden.\\  
In diesem Projekt war der Projektleiter oft mit dem Projektmanagement beschäftigt, deshalb hatte er gegenüber seinen Fachspezialisten teilweise einen grossen Wissensrückstand im technischen Bereich. Diese Wissenslücke machte sich in der Planung bemerkbar und beeinträchtigte den Planungsprozess. Zudem sollten die Sitzungen in Zukunft besser vorbereitet werden, anderenfalls ist es ein Zeitverlust. In Zukunft sollte der Projektleiter besser informiert sein. Eine negative Eigenschaft des Projektleiters waren die straffen Forderungen an sein Team. Diese können die Moral und die Motivation beeinträchtigen. Es kam zwar selten vor, doch dies sollte in Zukunft besser gehandhabt werden.\\ 
Die Fachinputs halfen dem Team Unklarheiten aufzudecken. Die Inputs interpretierte das Team als Kundenkontakt. In der Arbeitswelt ist es wichtig, dass das Projektteam einen regelmässigen Kontakt mit dem Kunden pflegt.\\
Zusammengefasst sind positive wie auch negative Ereignisse Erfahrungen, die dazu gehören. Das Team versucht dies Erkenntnisse in zukünftigen Projekten so gut als möglich umzusetzen. Doch die wichtigste Eigenschaft die das Team in Zukunft haben sollte, ist der Zusammenhalt und das gegenseitige Vertrauen.