Schlusswort 

Der Fortschritt im letzten Projekt war riesig. Ob schulische Projekt oder Firmenprojekte das Team konnte wertvolle Erfahrungen für die Zukunft sammeln. Mit den vielen Korrespondenz und dem zyklischen Abgaben, haben sich die sprachlichen Fähigkeiten im Team verbessert. Die Aufgabe als Projektleiter war ebenso anspruchsvoll wie die technischen Erarbeitungen oder die Umsetzung der Software. Der Job als Projektleiter brachte mehr Schreibaufwand mit sich. Mit den Statusberichte musste der aktuellen Stand zusammengefasst werden. Dies brachte die nötige Übersicht. Die Fachinputs halfen dem Team Unklarheiten aufzudecken. Die Inputs interpretierte das Team als Kundenkontakt. In der Arbeitswelt ist es wichtig, dass das Projektteam einen regelmässigen Kontakt mit dem Kunden pflegt.
Positive wie auch negative Ereignisse gehören zu den Erfahrungen. Sie werden in zukünftigen Projekten so gut als möglich umgesetzt. Doch die wichtigste Eigenschaft die eine Team in Zukunft haben sollte,ist der Zusammenhalt und das gegenseitige Vertrauen.

 