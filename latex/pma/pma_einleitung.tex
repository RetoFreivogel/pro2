Das Projekt handelt von der Regelungstechnik. Das Ziel ist es ein Programm in Java zu schreiben,
das die Schrittantwort der Regelung berechnen kann mit der Phasengangmethode von Jakob Zellweger
oder mit einer von ausgewählten Faustformeln und diese dann grafisch darstellt.\textcurrency 
Die Projektziele werden in drei Abschnitten definiert. Für die Eingabe und Ausgabe erfolgt die
Dateneingabe mit einer PTn-Schrittantwort und mit Tg-, Tu-, Ks-Werten, der Bereich des Überschwingens
oder des Phasenrandes ist wählbar, es kann ausgewählt werden zwischen PI und PID Regler, die
Schrittantwort der Regelung wird graphisch dargestellt und als Option wird die Lösung der Phasengangmethode
mit der Lösung der Faustformel verglichen. Für die Berechnungen wird das Matlabskript in Java
Code übersetzt, kann der Regler mit Faustformeln berechnet werden und wird die Berechnung im GUI
dargestellt. Zu den Allgemeinen Zielen gehören die Benutzerfreundlichkeit des GUI und die Dimensionierung
der Zellweger-Phasengangmethode.
Der Projektleiter organisiert das Projekt in eine Fachgruppe Elektrotechnik und eine Fachgruppe
Programmieren. Für den Informationsaustausch finden wöchentlich Sitzungen statt. Weiter kommuniziert
er über Mail- und Kurznachrichtendienste für die Sitzungseinladungen und um die Mitglieder schnell
und unkompliziert zu kontaktieren. Dem Projektleiter stehen Software zur Verfügung wie MS Project,
Word, Visio und Latex-Programm, die ihm helfen Informationen darzustellen und weiterzuverarbeiten.
Die erarbeiteten Dateien können dann über gemeinsam genutzte Plattformen wie Github für Code basierte
Programme, Sharepoint und Dropbox geteilt und vom ganzen Team eingesehen und genutzt werden.

Eine positive Erfahrung in diesem Projekt war, dass Vertrauen in die Fähigkeiten der Mitglieder beim
ausführen ihrer Arbeiten sehr wichtig sein kann. Hingegen war die Schätzung der Softwareplanung eine
schwierige Aufgabe und darum eine negative Erfahrung, die massiv unterschätzt wurde.

Der nachfolgende Bericht ist gegliedert in die Analyse der Erfahrung und eine Reflexion.