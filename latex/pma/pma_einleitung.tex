Eine Lösung mit geringem Aufwand einen PI- oder PID-Regler zu dimensionieren, war der Wunsch des Kunden Herr Niklaus. Herkömmliche Faustformeln sind zwar schnell, jedoch sehr ungenau. Die Aufgabe vom Projekt war es, die Phasengangmethode von Jakob Zellweger zu analysieren und in Software umzusetzen. Die Phasengangmethode ist eine graphische Lösung, welche bislang von Hand aus geführt wird. Die Aufgabe war eine Kombination von Elektrotechnik und Objekt orientiertes programmieren. Das Hauptziel war es ein benutzerfreundliches  \textit{Graphical User Interface} (kurz GUI)  zu erstellen. Das GUI sollte mindestens mit die Phasengangmethode von Zellweger umsetzten können. Optional wollte der Kunde die konventionellen Faustformeln für Vergleichszwecke in der Software integrieren haben. Eine graphische Auswertung definierte er als Wunschziel, wurde jedoch fester Bestandteil in diesem Projekt. Das Ziel, dass das Überschwingen in Prozent angegeben werden kann, definierte der Kunde in der Mitte des Projekts. Das Projektteam bestand aus sechs Personen. Der Projektleiter teilte das Team in kleine Arbeitsgruppen auf. So konnte zielgerichteter und präziser gearbeitet werden. Wöchentliche Sitzung nutze das Team um die Kommunikation aufrecht zu behalten. Das Team konnte über den Stand der Arbeiten oder Erfahrungen berichten. Zu dem nutze der Projektleiter diese Sitzungen, um die Aufgaben zu kontrollieren oder neue Aufgaben zu erteilen. Sitzungsprotokolle dienten als Gedächtnisstütze. Medien wie E-Mail oder Chat Dienste nutze er für Sitzungseinladungen oder spontane Änderungen. Für die Datensicherung wurden die Dienste von Sharepoint in Anspruch genommen. 
Arbeit und Organisation haben sich als positive Erfahrungen herausgestellt. Besonders wichtig war das wöchentliche Arbeiten am Projekt. Die Teammitglieder konnten sich fortlaufend austauschen und untereinander diskutieren. Zu dem wurde die Aufgaben, dank einer klaren Planung, gut verteilt. Sitzungen und  Auftragsklärung wurden hingegen als negative Erfahrungen wahrgenommen. Die Sitzungen waren und koordiniert. Dies führte zu lange Sitzungen. Zusätzlich hatte das Team lange bei der Analyse der Auftragsklärung. Dies führte zu Zeitknappheit, weshalb das Wunschziel Monte-Carlo Analyse nicht erreicht werden konnte.
Der nachfolgende Bericht ist in zwei Teile gegliedert. Die Analyse der Erfahrung zeigt Stichwortartig auf, welche positive und negative Erfahrungen gemacht wurden. Die Reflexion beschreibt davon je zwei positive und negative Erfahrungen. 