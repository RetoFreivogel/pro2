\documentclass{fhnwreport} %
\usepackage[ngerman]{babel}
\usepackage[T1]{fontenc}
\usepackage[latin1]{inputenc}
\usepackage{tikz}
\usepackage{amsmath}
\usetikzlibrary{arrows}
\usepackage{lmodern}   %Type1-Schriftart f�r nicht-englische Texte 

\title{%
  \textsc{Projekt 2}\\[2ex]
  \textsc{Disposition}}
\author{%
  \textsc{Team 1}}
\date{01.05.2015}

\begin{document}

\maketitle

% Titelbild
% (kann man nat�rlich auch mit Includegraphics machen)

\vfill

\textsc{%
\begin{tabbing}
Auftraggeber:  \=  Peter Niklaus \\[2ex]
Betreuer:  \=  Pascal Buchschacher, Anita Gertiser \\[2ex]
Experten:  \=  Peter Niklaus, Richard Gut \\[2ex]
Team:  \= Alexander Stocker \\ 
\= Claudius J�rg \\
\= Denis Stampfli \\
\= Martin Moser \\
\= Reto Freivogel \\
\= \= Yohannes Measho \\[2ex]
Studiengang: \= Elektro- und Informationstechnik
\end{tabbing}}

\vfill
\hbox{}

\clearpage

%\tableofcontents
%\newpage

\section{Einleitung}
\section{Reglertyp}
\subsection{Typ-PI}
\subsection{Typ-PID}
\subsection{�bertragungsfunktion der Regler}
\section{L�sungskonzept}
\subsection{Phasen- und Amplitudendiadramm}
\subsection{Identifikation Tu,Tg und Ks  von Diagrammen}
\section{Java Software}
\subsection{Klassendiagramm}
\subsection{Beschreibung der Software}
\subsection{Benutzerschnittstelle}
\subsection{Klassen}
\subsubsection{GUI Klassen}
\subsubsection{Model Klassen}
\subsubsection{View Klassen}
\subsubsection{Controller Klassen}
\section{Test Matlabs}
\subsection{Schrittantwort}
\subsubsection{�bertragungsfunktion}
\subsubsection{Wendetangent}
\subsection{Dimensionierung mit Faustformeln}
\subsection{Dimensionierung mit  Phasengangmethode}
\subsubsection{Amplitudengang der Strecke}
\subsubsection{Phasengang der Strecke}
\section{Schlusswort}
\subsection{Schrittantwort}
\subsubsection{�bertragungsfunktion}
\subsubsection{Wendetangent}
\subsection{Dimensionierung mit Faustformeln}
\subsection{Dimensionierung mit  Phasengangmethode}
\subsubsection{Amplitudengang der Strecke}
\subsubsection{Phasengang der Strecke}
\section{Literaturverzeichnis}


\end{document}

