\section{Schlussfolgerung}
Das Ergebnis am Ende dieses Projektes ist eine Software welche auf dem MVC-Pattern aufgebaut ist. Die Streckenparameter können in der Software eingegeben werden und es resultieren die Parameter des Reglers mit der Schrittantwort der Regelungsstrecke. Die Schrittantwort der Regelung wird unmittelbar vermasst und die Vermassungswerte werden im Analyse Panel dargestellt. Damit verschiedene Regler verglichen werden können, besteht die Möglichkeit, mehrere Plots zu generieren. Das Ziel, dass die Software einfach zu bedienen ung somit benutzerfreundlich sein soll, haben wir mit einer übersichtlichen Darstellung und geeigneten Tastenkombinationen (Shortcuts) realisiert.\\
Im Grossen und Ganzen haben wir die gesetzten Ziele erreicht. Insbesondere die Soll-Ziele wurden alle erreicht.. Nicht erreicht wurde das Wunschziel der Monte-Carlo-Analyse. Die Idee war, durch zufällige Wahl der Eingabewerte in einem gewissen Prozentbereich, verschiedenste Regelungen zu berechnen und somit den maximalen Fehler optisch darstellen zu können. Aus zeitlichen Gründen konnte dieses Ziel jedoch nicht weiter verfolgt und realisiert werden.