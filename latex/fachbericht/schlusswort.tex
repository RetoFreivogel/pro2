\section{Schlusswort}
Ein Software welche auf dem MVC-Pattern aufgebaut ist das Ergebnis am Ende des Projektes. Die Streckenparameter können in der Software eingegeben werden und es resultieren die Parameter des Reglers mit der Schrittantwort der Regelungsstrecke. Die Schrittantwort der Regelung wird unmittelbar vermasst und die Vermassungswerte werden im Analyse Panel dargestellt. Da die Reglerparamter über die Sanimethode berechnet werden, können die Zeiten der Sanimethode ausgewählt werden. Damit verschiedene Regler verglichen werden können, besteht die Möglichkeit, mehrere Plots zu generieren. Das Ziel einfache Bedienung haben wir mit einer übersichtlichen Darstellung und geeigneten Tastenkombinationen realisiert.
Im Grossen und Ganzen haben wir die gesetzten Ziele erreicht. Die Idee hinter der Monte-Carlo Analyse war es, eine Regelung zu berechnen, welche die prozentuale Abweichung graphisch darstellte. Die Eingabewerte der Strecke sollten dabei in einem Range von fünf Prozent tausendmal berechnet werden. Doch aus zeitlichen Gründen konnte dieses Ziel nicht realisiert werden.