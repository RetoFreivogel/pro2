\documentclass{fhnwreport}
\usepackage[ngerman]{babel}
\usepackage[T1]{fontenc}
\usepackage[latin1]{inputenc}
\usepackage{tikz}
\usepackage{amsmath}
\usetikzlibrary{arrows}
\usepackage{lmodern}
\usepackage[final]{pdfpages}
\usepackage{graphicx}

\title{%
  \textsc{Projekt 2}\\[2ex]
  \textsc{Fachbericht}}
\author{%
  \textsc{Team 1}}
\date{%
  \textsc{10.06.2015}}

%\textit{} % italics
%\textbf{} % bold
%\texttt{} % typewriter style
%\textsf{} % sans-serif
%\textsc{} % all capital letters

\begin{document}
\maketitle

\vfill

\textsc{%
\begin{tabbing}
Auftraggeber: \hspace{4em} \=  Peter Niklaus \\[2ex]
Betreuer:  \>  Pascal Buchschacher, Anita Gertiser \\[2ex]
Experten:  \>  Peter Niklaus, Richard Gut \\[2ex]
Team:  \> Alexander Stocker \\ 
\> Claudius J�rg \\
\> Denis Stampfli \\
\> Martin Moser \\
\> Reto Freivogel \\
\> Yohannes Measho \\ [2ex]
Studiengang: \> Elektro- und Informationstechnik
\end{tabbing}}

\clearpage

\tableofcontents
\newpage
\section{Einleitung}
\section{Theoretische Grundlagen}
\subsection{Schrittantwort}
\subsubsection{�bertragungsfunktion}
\subsubsection{Wendetangent}
\subsection{Dimensionierung mit Faustformeln}
\subsection{Dimensionierung mit  Phasengangmethode}
\subsubsection{Amplitudengang der Strecke}
\subsubsection{Phasengang der Strecke}
\section{Reglertyp}
\subsection{Typ-PI}
\subsection{Typ-PID}
\subsection{�bertragungsfunktion der Regler}
\section{L�sungskonzept}
\subsection{Phasen- und Amplitudendiadramm}
\subsection{Identifikation Tu,Tg und Ks  von Diagrammen}
\section{Java-Software}
\subsection{Klassendiagramm}
\subsection{Beschreibung der Software}
\subsection{Benutzerschnittstelle}
\subsection{Klassen}
\subsubsection{GUI Klassen}
\subsubsection{Model Klassen}
\subsubsection{View Klassen}
\subsubsection{Controller Klassen}
\section{Test anhand Matlabs}
\section{Schlusswort}
\section{Literaturvezeichnis}

\end{document}

